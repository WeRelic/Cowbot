\documentclass{article}
\usepackage{Macros}

\newcommand{\class}[1]{\section{class \texttt{#1}}}
\newcommand{\attribute}[1]{\subsection{\texttt{self.#1}}}
\newcommand{\method}[1]{\subsection{\texttt{self.#1}}}
\newcommand{\function}[1]{\section{\texttt{#1}}}
\newcommand{\argumenta}[1]{\subsection{\texttt{#1}}}
\newcommand{\argumentb}[1]{\subsubsection{\texttt{#1}}}


\begin{document}
\begin{flushleft}

\begin{center}
  Cowbot Documentation: Pathing.ArcLineArc.py
\end{center}

TODO: Clean up \texttt{\_\_init()\_\_} so it's not as much a wall of code.


\class{ArcLineArc}
      {

        \argumenta{start}
                  {
                    \texttt{Vec3}.  The starting point for the path.
                  }
        \argumenta{end}
                  {
                    \texttt{Vec3}.  The ending point for the path.                    
                  }
        \argumenta{start\_tangent}
                  {
                    Nonzero \texttt{Vec3}.  The tangent to the path at \texttt{start}.
                  }
        \argumenta{end\_tangent}
                  {
                    Nonzero \texttt{Vec3}.  The tangent to the path at \texttt{end}.
                  }
        \argumenta{radius1}
                  {
                    Float.  The signed radius of the first arc.  Positive for CW, negative for CCW.
                  }
        \argumenta{radius2}
                  {
                    Float.  The signed radius of the second arc.  Positive for CW, negative for CCW.
                  }
        \argumenta{current\_state}
                  {
                    \texttt{CarState}.  The current state of our car, probably from \texttt{game\_info.me}.
                  }


        \attribute{start\_normal}
                  {
                    \texttt{Vec3}.  The normal vector to the start tangent.  Facing ????
                  }
        \attribute{end\_normal}
                  {
                    \texttt{Vec3}.  The normal vector to the end tangent.  Facing ????
                  }
        \attribute{center1}
                  {
                    \texttt{Vec3}. The location of the center of the circle the first arc lies along.
                  }
        \attribute{center2}
                  {
                    \texttt{Vec3}. The location of the center of the circle the second arc lies along.
                  }
        \attribute{transition1}
                  {
                    \texttt{Vec3}. The location of the transition from the first arc to the line segment.
                  }
        \attribute{transition2}
                  {
                    \texttt{Vec3}. The location of the transition from the line segment to the second arc.
                  }
        \attribute{is\_valid}
                  {
                   Boolean.  True when the ArcLineArc is valid, using conditions added as seen fit.  TODO: Add checks that we don't leave the floor of the stadium.
                  }

        \method{find\_lengths()}
               {
                 Returns three floats: The length of the first arc, the length of the line segment, and the length of the second arc.
               }
        \method{draw\_path()}
               {
                 Uses the RLBot renderer to draw the ArcLineArc path on screen.  Disable for tournament versions.
               }
               



















      }
      
         









  
  





\end{flushleft}
\end{document}

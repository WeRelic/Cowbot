\documentclass{article}
\usepackage{Macros}

\newcommand{\class}[1]{\section{class \texttt{#1}}}
\newcommand{\attribute}[1]{\subsection{\texttt{self.#1}}}
\newcommand{\method}[1]{\subsection{\texttt{self.#1}}}
\newcommand{\function}[1]{\section{\texttt{#1}}}
\newcommand{\argumenta}[1]{\subsection{\texttt{#1}}}
\newcommand{\argumentb}[1]{\subsubsection{\texttt{#1}}}


\begin{document}
\begin{flushleft}

\begin{center}
  Cowbot Documentation: Miscellaneous.py
\end{center}


\function{cap\_magnitude(x, magnitude = 1)}
         {
           Returns \texttt{sgn(x)*max(abs(x),magnitude)}.  The output and both arguments are floats.
                              
         }
         



\function{rotate\_to\_range(theta, interval)}
         {
           \argumenta{theta}
                    {
                      Float.
                    }
           \argumenta{interval}
                    {
                      List or tuple of two floats.  \texttt{interval[0]} should be strictly less than \texttt{interval[1]}.
                    }

           This function is primarily used to find an angle equivalent to \texttt{theta} but is between $-\pi$ and $\pi$.  Add and subtract multiples of the length of \texttt{interval} to \texttt{theta} until \texttt{theta} is between \texttt{interval[0]} and \texttt{interval[1]}. Return the final theta.
         }




\function{car\_coordinates\_2d(current\_state, direction)}
         {
           Throws out the $z$-component and rotates the $x$ and $y$ components to the car-basis.  Returns a \texttt{Vec3} with zero $z$-component.

           \argumenta{current\_state}
                     {
                       \texttt{CarState}.  The state of the car we're working with.
                     }
           \argumenta{direction}
                     {
                       \texttt{Vec3}.  The direction vector from the car to the target.
                     }
         }



\function{angles\_are\_close(angle1, angle2, epsilon)}
         {
           Returns a Boolean, True if \texttt{angle1} (float) and \texttt{angle2} (float) are within \texttt{epsilon} (float) of each other.
         }





\function{left\_or\_right(current\_state, target\_pos)}
         {
           Check if the car should turn left or right to face towards the target.  Returns +1 for right and -1 for left.

           \argumenta{current\_state}
                     {
                       \texttt{CarState}.  The current state of the car.
                     }
                     


           \argumenta{target\_pos}
                     {
                       \texttt{Vec3}. The point on the field we are trying to point towards.
                     }
                     
         }
         



\function{rot\_to\_mat3(rot}
         {
           Takes \texttt{rot}, an \texttt{Orientation} object, and returns the corresponding RLU \texttt{mat3} object.
         }




\function{pyr\_to\_matrix(pyr)}
         {
           TODO: Change all ``pyr'' notation to ``ypr'' to match the convention used by RL.\\
           Takes an Euler angle orientation (pitch, yaw, roll) and returns the orientation matrix \texttt{[front, left, up]}.
         }




\function{Vec3\_to\_Vector3(vector)}
         {
           Takes a \texttt{Vec3} (CowBot) and returns the corresponding \texttt{Vector3} (framework).
         }


         
         
\function{Vec3\_to\_vec3(vector)}
         {
           Takes a \texttt{Vec3} (CowBot) and returns the corresponding \texttt{vec3} (RLU).
         }

         

\function{vec3\_to\_Vec3(vector)}
         {
           Takes a \texttt{vec3} (RLU) and returns the corresponding \texttt{Vec3} (CowBot).
         }






\function{is\_in\_map(location)}      
         {
           Takes \texttt{location} (Vec3) and returns a Boolean, True if \texttt{location} is inside the game map.  Rudimentary for now, can definitly be improved over time.
         }



\function{angle\_to(target, start, initial\_angle)}
         {
           Takes a target location \texttt{target} (\texttt{Vec3}), a starting location \texttt{start} (\texttt{Vec3}), and a starting yaw \texttt{initial\_angle} (float between $-\pi$ and $\pi$).\\
           Returns the angle between the initial yaw and the angle needed to face the target.
         }





\function{min\_radius(speed)}
         {
           Returns the minimum radius (float) possible given an input speed (float).  Comes from Chip's notes on ground handling.  Data was taken for an Octane, turns in a plank body will likely be slightly wider.
         }



\function{max\_speed(radius)}
         {
           Returns the maximum speed (float) possible given an input radius (float).  Comes from Chip's notes on ground handling.  Data was taken for an Octane, turns in a plank body will likely be slightly wider.
         }




























  
  





\end{flushleft}
\end{document}
